\documentclass[class=scrreprt,11pt]{standalone}
%
\usepackage{tikz}
\usetikzlibrary{trees}
\begin{document}
\tikzstyle{every node}=[anchor=west,minimum width={2.2cm},font=\normalsize]
\tikzstyle{folder}=[draw=black,thick]
\tikzstyle{selected}=[draw=red,thick,fill=red!30]
\tikzstyle{root}=[draw=green,thick,fill=green!30]
\tikzstyle{optional}=[dashed,thick,fill=gray!50]
\begin{tikzpicture}[%
  grow via three points={one child at (0.5,-0.7) and
  two children at (0.5,-0.7) and (0.5,-1.4)},
  edge from parent path={(\tikzparentnode.south) |- (\tikzchildnode.west)},
  edge from parent/.style={draw,very thick,-latex}]
  \node [root] {annika}
    child { node [selected] {Bachelor\_Arbeit}
      child { node [selected] {Beobachtungen}
        child[edge from parent/.style={draw,very thick}] { node [folder] {processed}}
        child[edge from parent/.style={draw,very thick}] { node [folder] {raw}}
      }
    }
    child [missing] {}
    child [missing] {}
    child [missing] {}
    child[edge from parent/.style={draw,very thick}] { node [folder] {figuren}}
    child[edge from parent/.style={draw,very thick}] { node {\dots}};
  %
  \node[anchor=center,align=center,yshift=2em]
      at (current bounding box.north)
      {Im Verzeichnisbaum nach \textit{unten} gehen};
\end{tikzpicture}
\end{document}
