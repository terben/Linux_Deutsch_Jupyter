\documentclass[border=1in]{standalone}
%
% Figur um Unix-Pipes zu visualisieren
\usepackage{tikz}
\usetikzlibrary{positioning,shapes.symbols}
%
\begin{document}

%
% Stil für die Kommandozeilen:
\tikzset{commline/.style = {font = \bf\LARGE,
    text depth = depth("g"),
    text height = height("F")}}
%
% Stil für Erklärungskästen unter den
% Hommandozeilen
\tikzset{boxexp/.style = {fill,
    font = \bf\LARGE,
    text = white,
    rounded corners, align=center}}
%
% Stil für Pfeile zwischen Erklärungskästchen
\tikzset{arrowexp/.style = {fill,
    signal,
    text=white,
    minimum height = 3ex,
    minimum width = 2em}}

\begin{tikzpicture}

% Erstes Beispiel: simples wc
\path (0,0) node[commline]
      (a) {\$};
\node[commline, color=red, right=0.1cm of a]
      (b) {wc -l *.txt};
      
\node[boxexp, fill = red, below=0.5cm of b]
      (c) {wc -l *.txt};
\node[arrowexp,
      signal to = east,
      fill = red,
      right=-0.05cm of c]
      (d) {\ out};
\node[arrowexp,
      signal from = west,
      signal to = nowhere,
      fill=black,
      right=-0.05cm of d]
      (e) {};
\node[boxexp, fill=black,
      right=-0.05cm of e]
      {Ausgabe \\ auf \\ Bildschirm };

% Zweites Beispiel:
% wc mit Umleitung in Datei
\path (0,-3.5) node[commline]
      (a1) {\$};
\node[commline, color = red, right = 0.1cm of a1]
      (b1) {wc -l *.txt};
\node[commline, font=\boldmath\LARGE,
      color = black, right = 0.1cm of b1]
      (c1) {\(>\)};
\node[commline,color = blue, right = 0.1cm of c1]
      (d1) {laengen.txt};
      
\node[boxexp, fill = red,
      below = 1.0cm of b1]
      (e1) {wc -l *.txt};
\node[arrowexp, signal to = east, fill = red,
      right=-0.05cm of e1]
      (f1) {\ out};
\node[arrowexp, signal from = west, signal to = nowhere,
      fill=blue,
      right=-0.05cm of f1]
      (g1) {};
\node[boxexp, fill=blue,
      right=-0.05cm of g1]
      {Ausgabeumleitung \\ in Datei \\ laengen.txt };

% Drittes Beispiel:
% wc mit zwei Pipes und Bildschirmausgabe:
\path (0,-7.5) node[commline]
      (a2) {\$};
\node[commline,color=red, right = 0.1cm of a2]
      (b2) {wc -l *.txt};
\node[commline,right=0.1cm of b2]
      (c2) {\textbar};
\node[commline,color=orange,right = 0.1cm of c2]
      (d2) {sort -n};
\node[commline,right=0.1cm of d2]
      (e2) {\textbar};
\node[commline,color=violet,right = 0.1cm of e2]
      (f2) {head -n 1};

\node[boxexp, fill = red,
      below=1.0cm of b2]
      (g2) {wc -l *.txt};
\node[arrowexp, signal to=east, fill=red,
      right=-0.05cm of g2]
      (h2) {\ out};
\node[arrowexp, signal from = west, signal to = nowhere,
      fill=orange, right=-0.05cm of h2]
      (i2) {in\ };
\node[boxexp, fill=orange,
      right=-0.0cm of i2]
      (j2) {sort -n};
\node[arrowexp, signal to=east, fill=orange,
      right=-0.05cm of j2]
      (k2) {\ out};            
\node[arrowexp, signal from=west, signal to=nowhere,
      fill=violet, right=-0.05cm of k2]
      (l2) {in\ };
\node[boxexp, fill=violet,
      right=-0.05cm of l2]
      (m2) {head -n 1};
\node[arrowexp, signal to=east, fill=violet,
      right=-0.05cm of m2]
      (n2) {\ out};    
\node[arrowexp, signal from=west, signal to=nowhere,
      fill=black, right=-0.05cm of n2]
      (o2) {};
\node[boxexp, fill=black,
      right=-0.05cm of o2]
      {Ausgabe \\ auf \\ Bildschirm };              
      
\end{tikzpicture}
\end{document}
